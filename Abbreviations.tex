\makeglossaries
% List acronyms you would like to have appear in the List of Abbreviations. 
% Each acronym is in the format \newacronym{tag}{acronym}{long}
% The "tag" field is used when defining the acronym in the main text. The "acronym" field  contains the abbreviated form, and the "log" field contains the unabbreviated definition.
% Note: in order for the acronym to appear in the List of Acronyms, it must be tagged in the main text.
% To tag the acronym in the main text, you can write something like:  \acrlong{dna} (abbreviated \acrshort{dna}) is the molecule that carries genetic information for the development and functioning of an organism.

\newacronym{dna}{DNA}{Deoxyribonucleic acid}

\newacronym{rna}{RNA}{Ribonucleic acid}