\begin{appendices}

% Some Table of Contents entry formatting
\addtocontents{toc}{\protect\renewcommand{\protect\cftchappresnum}{\appendixname\space}}
\addtocontents{toc}{\protect\renewcommand{\protect\cftchapnumwidth}{6em}}

% Begin individual appendices, separated as chapters

\chapter{First Chapter of Appendix}

In academic or written work, an appendix refers to supplementary material or additional information that's added at the end of a book, document, or research paper. This can include charts, graphs, tables, detailed information, or other supporting data that isn't essential to the main body of the work but could provide clarification or further details for interested readers.


\chapter{Second Chapter of Appendix}

The human body has an organ called the appendix. It's a small, tube-like pouch attached to the large intestine, usually located in the lower right side of the abdomen. While the exact function of the appendix isn't entirely clear, it's believed to play a role in the immune system, specifically in younger individuals. In some cases, the appendix can become inflamed or infected, leading to a condition called appendicitis, which often requires surgical removal. 

\end{appendices}

